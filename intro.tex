\section{Introduction}
% \subsection{Quantum Chromodynamics and Quark-Gluon Plasma}
In 1947, half a century after J.J. Thomson discovered the electron, Cecil Powell discovered the pion\cite{occhialini_nuclear_1947}. In the years immediately following, a large number of seemingly new fundamental particle states were discovered. In 1964 Gell-Mann and George Zweig proposed the quark model, suggesting that the recently discovered particles, such as pions, were not fundamental but instead composed of constituents called quarks. The quark model continued to develop into the current theory of quantum chromodynamics(QCD), a theory describing the fundamental interactions between matter particles called quarks and force carriers called gluons. 

The characteristics of QED cause the effective interaction strength of the photon ($\alpha$) to decrease with increasing distance and approach the asymptotic value of $\alpha\approx1/137$ at large distances. However, in QCD the gluon carries color charge and can have self-interactions. These characteristics of QCD cause the effective interaction strength of the gluon ($\alpha_s$) to increase with distance. This causes quarks and gluons to become confined to bound states at distances larger than about 1 fermi and is referred to as `confinement'. However, at smaller and smaller distance scales, the effective interaction strength of QCD decreases until the quarks and gluons are effectively non-interacting free particles. 

	\begin{figure}
		\centering 
		\begin{subfigure}[b]{0.6\textwidth} 
			\includegraphics[width=\textwidth]{qcd_phase.png} 
			\label{fig:qcd_phase_diagram} 
		\end{subfigure} 
		\begin{subfigure}[b]{0.39\textwidth} 
			\includegraphics[width=\textwidth]{asq-2009.pdf} 
			\label{fig:qcd_running_coupling} 
		\end{subfigure} 
		\caption{ A schematic diagram of the QCD phase space showing the conjectured critical point and phase transition into QGP (left)\cite{bicudo_chiral_2011}. The value of $\alpha_{s}$ as a function of the momentum exchange (Q) as predicted by QCD and measured from various types of experiments (right) \cite{bethke_2009_2009}. } 
	\end{figure}

This effect, called `asymptotic freedom', was discovered by David Gross and Frank Wilczek in 1973\cite{gross_ultraviolet_1973}. Shortly after, in 1977, Bohr, Henrik and Nielsen predicted that asymptotic freedom should lead to a de-confined state of QCD matter at high temperatures and densities\cite{bohr_hadron_1977}. Discovery of and classification of this de-confined state of matter, called Quark-Gluon Plasma (QGP), has been a main focus of high energy nuclear physics ever since. Heavy ion collisions at facilities like the Relativistic Heavy Ion Collider (RHIC) and the Large Hadron Collider (LHC) are expected to produce systems that reach sufficiently high temperature and density to enter the QGP phase\cite{bohr_hadron_1977}.


