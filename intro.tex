\section{Introduction}
In 1897 J.J. Thomas discovered the electron\cite{}. Over the next 40 years the extremely successful theory of quantum electrodynamics was developed to describe electromagnetic interactions. Half a century after the discovery of the electron Cecil Powell discovered the pion\cite{}. Over the next several years experimentalists aided by recent technological improvements discovered a large number of new particles. The proliferation of new particle states suggested that there was some underlying structure to these particle states. In 1964 Gell-Mann and George Zweig proposed the quark model\cite{}, suggesting that the recently discovered particles, such as pions, were not fundamental but instead composed of constituents called quarks. 

QCD differs from QED in a few very imp


In 1947, half a century after J.J. Thomson discovered the electron, Cecil Powell discovered the pion\cite{}\cite{}. In the years immediately following, a large number of seemingly new fundamental particle states were discovered. In 1964 Gell-Mann and George Zweig proposed the quark model, suggesting that the recently discovered particles, such as pions, were not fundamental but instead composed of constituents called quarks\cite{}. The quark model continued to develop into the current theory of quantum chromodynamics(QCD), a theory describing the fundamental interactions between quarks and gluons. 



\subsection{Heavy Ion Collisions}

The exploration of the QCD phase diagram requires the preparation of systems with extreme temperatures and densities.  
