\section{The Solenoidal Tracker at RHIC}
The Solenoidal Tracker At RHIC (STAR) is one of several experimental apparatus at BNL used for studying the systems produced in heavy ion collisions. The STAR detector, shown in Figure \ref{fig:star}, consists of several sub-detectors that each perform a specialized function. Only three of the detectors are important for the work being presented: the Time Projection Chamber (TPC), the barrel Time Of Flight detector and the Vertex Position Detector (VPD). 

\subsection{ Time Projection Chamber }
The TPC is the primary detector in STAR. It provides charged particle tracking as well as momentum and specific ionization energy loss (dE/dx) measurement. A schematic diagram of it's structure is shown in Figure \ref{fig:TPC}. It is a metal cylinder 400 cm in diameter and 420 cm long. The central membrane, the cathode, is at 28 kV with respect to the anodes at either end. The chamber is filled with P10 gas, a mixture of $\sim$10\% methane and $\sim$90\% argon\cite{anderson_star_2003}. 

Charged particles passing through the TPC's gas cause the gas to become ionized along the particle's trajectory. The potential difference between the anodes and the cathode causes the charge to drift towards pick-up pads on either sides of the TPC. Read-out electronics then record the location of the charge and the time of their arrival. This data is used by specialized software to reconstruct the particle trajectory, momentum and dE/dx.


\subsection{ Time of Flight System }
The purpose of the STAR Time Of Flight (TOF) system is to provide precise timing of particle trajectories. The TOF system is essentially a specialized stopwatch with dedicated components for measuring the start-time and stop-time of particle trajectories.

The VPD is dedicated to providing precise event-by-event start-time measurements. The VPD is a set of two detectors on either side of the interaction point (one one the east and one one the west) along the beam-line. Each detector assembly consists of 19 photomultiplier tubes capable of detecting charged particles that are minimally deflected in the plane transverse to the beam-line. The coincidence between the two detectors along with the precise signal timing is used to estimate the start-time of the initial heavy ion interaction to a resolution of $\sim$85 picoseconds in gold on gold collisions at RHIC top energies\cite{llope_star_2014}.

The student has been active in the maintenance and support of the TOF system since joining Rice University. The student setup the VPD in January 2015 for use the following year. The student has also been responsible for performing the electronics and software calibrations needed to achieve the expected time resolution for data-taking runs since 2013.