\section{Study of the Nuclear Modification Factor}
The nuclear modification factor $R_{AA}$ is the ratio of particle production in a heavy ion collisions to the appropriately scaled particle production of individual proton on proton (pp) collisions. If a heavy ion collision is simply a superposition of pp collisions then the $R_{AA}$ will be equal to 1.0. However, if there is a net enhancement or suppression of particle production in heavy ion collisions compared to pp collisions then the $R_{AA}$ will be above or below 1.0 respectively. When pp reference data is not available, the nuclear modification factor can be measured using the modified ratio, $R_{CP}$. The $R_{CP}$ is a proxy to the $R_{AA}$ measurement and is defined as:

\begin{equation}
	R_{CP} = \frac{ \langle N_{coll} \rangle_{Peripheral} } {  \langle N_{coll} \rangle_{Central} } \frac{ d^2 N^{Central} / dp_{T}dy } { d^2 N^{Peripheral} / dp_{T}dy }
	\label{eq:r_cp}
\end{equation}

Where $Central$ refers to events in which the ions collided with a small impact parameter ($b$), $Peripheral$ refers to events in which the ions collided with a larger $b$, $\langle N_{coll} \rangle $ is the appropriate scaling factor discussed in Section \ref{sec:mc_glauber} and $dN^2/dp_T dy$ is the invariant particle yield measured as a function of the momentum transverse to the beam-line ($p_T$) and the rapidity ($y$), a measure of the particle's velocity in the direction of the beam-line. 

The $R_{CP}$ has been previously measured at STAR in Au$+$Au collisions at $\sqrt{s_{NN}}=$200GeV. At this energy the $R_{CP}$ was found to be significantly less than 1.0 for $p_T \geq \sim$3.0 GeV/c \cite{}. This suppression of particle production at high $p_T$ has been attributed to the loss of energy of particles traveling through and interact with a QGP-like medium. The work presented here is a study of the nuclear modification factor $R_{CP}$ for identified $\pi^{+/-}$, K$^{+/-}$, $p$ and $\bar{p}$ at $\sqrt{s_{NN}}=14.5$ GeV/c. The purpose of this analysis is to determine if this signature attributed to the QGP is visible in data collected at $\sqrt{s_{NN}}=14.5$ GeV.





In 2010, RHIC began a specialized program called the Beam Energy Scan (BES) phase I with the purpose of scanning the QCD phase diagram (See Figure \ref{fig:qcd_phase}). The primary purpose of this program is threefold: 1) to find a conjectured critical point, 2) to search for a first order phase transition, and 3) to study the turn-off of signatures attributed to the formation of a QGP. The study presented here is of the $R_{CP}$ for the data collected during the BES. The student has developed an analysis capable of 

In order to properly measure $R_{AA}$, data must be collected for pp collisions at the same conditions as the heavy ion collisions of interest. For the BES it was not possible to collect corresponding pp data. For this reason a proxy to the $R_{AA}$ ratio is needed that gives similar sensitivity to the affects of the medium on particle production. We can define a similar ratio, $R_{CP}$, that is the ratio of appropriately scaled yields in events with a small impact parameters (b) over appropriately scaled yields in events with a larger impact parameter. This ratio was measured at STAR in Au$+$Au collisions at $\sqrt{s_{NN}}=$200GeV. The $R_{CP}$ was measured as a function of the momentum in the plane transverse to the beam-line ($p_T$) and found to be significantly less than 1.0 for $p_T \geq $ 3.0 GeV/c \cite{}. This suppression of particle production at high $p_T$ has been attributed to the loss of energy of particles traveling through and interact with a QGP-like medium.

