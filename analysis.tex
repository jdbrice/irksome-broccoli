\section{Study of the Nuclear Modification Factor}
	The nuclear modification factor $R_{AA}$ is the ratio of particle production in heavy ion collisions to the appropriately scaled particle production of individual proton on proton (pp) collisions. If a heavy ion collision is simply a superposition of pp collisions then the $R_{AA}$ will be equal to 1.0. However, if there is a net enhancement or suppression of particle production in heavy ion collisions compared to pp collisions then the $R_{AA}$ will be above or below 1.0, respectively. When pp reference data is not available, the nuclear modification factor can be measured using the modified ratio, $R_{CP}$. The $R_{CP}$ is a proxy to the $R_{AA}$ measurement and is defined as:

	\begin{equation}
		R_{CP} = \frac{ \langle N_{coll} \rangle_{Peripheral} } {  \langle N_{coll} \rangle_{Central} } \frac{ d^2 N^{Central} / dp_{T}dy } { d^2 N^{Peripheral} / dp_{T}dy }
		\label{eq:r_cp}
	\end{equation}

	Where $Central$ refers to events in which the ions collided with a small impact parameter ($b$), $Peripheral$ refers to events in which the ions collided with a larger $b$, $\langle N_{coll} \rangle $ is the appropriate scaling factor discussed in Section \ref{sec:mc_glauber} and $dN^2/dp_T dy$ is the invariant particle yield measured as a function of the momentum transverse to the beam-line ($p_T$) and the rapidity ($y$), a measure of the particle's velocity in the direction of the beam-line. 

	The $R_{CP}$ has been previously measured at STAR in Au$+$Au collisions at $\sqrt{s_{NN}}=$ 200 GeV. At this energy the $R_{CP}$ was found to be significantly less than 1.0 for $p_T \geq \sim$3.0 GeV/c \cite{}. This suppression of particle production at high $p_T$ has been attributed to the loss of energy of particles traveling through and interact with a QGP-like medium. The work presented here is a study of the nuclear modification factor $R_{CP}$ for identified $\pi^{+/-}$, $p$ and $\bar{p}$ at $\sqrt{s_{NN}}=14.5$ GeV/c. The purpose of this analysis is to determine if this signature attributed to the QGP is visible in data collected at $\sqrt{s_{NN}}=14.5$ GeV.


\subsection{ Centrality and Scale Factor Determination }
	\label{sec:mc_glauber}

	It is impossible to directly measure the impact parameter of heavy ion collisions at RHIC. For this reason we employ an experimentally measurable observable, the event centrality, that is statistically related to the impact parameter. 


	Each beam of ions in typical RHIC fill consists of more than 1$\times 10^9$ ions that are accelerate to a large fraction of the speed of light. A collision occurs when a pair of ions, one from each passing cloud, interacts with one another. These collisions occur with a random uniform impact parameter.

	The scaling factor introduced in Eq. \ref{eq:r_cp} is the average number of binary nucleon-nucleon collisions for the given event class. The $N_{coll}$ for a pp collision is defined to be 1. In general, $N_{coll}$ for a heavy ion collision is related to the impact parameter of the collision. Events with a small impact parameter have, on average, more nucleons colliding and therefore a larger $N_{coll}$, while events with a larger impact parameter will have a smaller $N_{coll}$. Since the impact parameter cannot be directly measured in experiments, the events are classified using the multiplicity of charged particles produced in the event. A Monte Carlo (MC) Glauber simulation is then used to statistically relate the experimentally measured multiplicities to the impact parameter and $N_{coll}$. After this the 

	While the full outline of the Glauber simulation is beyond the scope of this paper, the student can provide further details if they are requested. The student conducted this study, produced new simulation code when necessary, and provided the event classification and related information to the entire STAR collaboration for use with the $\sqrt{s_{NN}}=14.5$ GeV data. shout out Mohammad

	\begin{figure}[ht]
		\centering 
		
		\includegraphics[width=\textwidth]{test_auau_2a.pdf} 
		\label{fig:glauber_auau} 

		\caption{ An example of Monte Carlo Glauber showing the nucleon positions of two gold nuclei that collide. The nucleons that participate in the collision are highlighted with thicker lines \cite{loizides_improved_2015-1}.  } 
	\end{figure}

\subsection{ Particle Identification}
	The primary challenge in measuring the $R_{CP}$ for identified particles is extracting exclusive particle yields from the tracks in each event. As mentioned in Section \ref{sec:star}, the TPC and TOF detectors are used in this analysis to identify particle species. The TPC provides dE/dx and the TOF provides 1/$\beta$ as seen in Fig. \ref{fig:typical_pid}. These quantities depend on the particle mass and are therefore useful for separating particle species. 

	\begin{figure}
		\centering 
		\begin{subfigure}[b]{0.49\textwidth} 
			\includegraphics[width=\textwidth]{typical_dedx.pdf} 
			\label{fig:typical_dEdx} 
		\end{subfigure} 
		\begin{subfigure}[b]{0.49\textwidth} 
			\includegraphics[width=\textwidth]{typical_onebeta.pdf} 
			\label{fig:typical_onebeta} 
		\end{subfigure}
		\caption{ \label{fig:typical_pid} The measured ln(dE/dx) versus $p_T \times $charge from the TPC (left). The measured velocity expressed at 1/$\beta$ versus $p_T \times$charge from the TOF (right). The bands correspond to particles of different masses. } 
	\end{figure}

	In this analysis the student developed a novel method of combining these two observables to maximize the separation power and stability of the yield extraction. For this procedure the measured dE/dx and 1/$\beta$ are first re-weighted to account for poor detector calibrations, momentum resolution effects, and for normalization. The measured values are transformed using :
	\begin{equation}
		z^{\prime} = \Sigma_{s} \left[ w_{s}(z) \times \left( z + \mu^{s}(\langle p \rangle ) - \mu^{s}(p) \right) \right] - \mu^{c}(\langle p \rangle)
	\end{equation}

	where $c$ is the species to center around, $s$ is the species index ( e.g. $\pi$, K, p ), $w_{s}$ is a weight given to each species as a function of the measurement, $p$ is the track momentum in GeV/c, $\langle p \rangle$ is the average momentum in the given $p_T$ and $y$ range, $\mu^{s}(p)$ is the theoretical mean for the measurement at the given momentum value, and $z$ is the measured value, either dE/dx or 1/$\beta$. These two variables are then used simultaneously in a $\chi^2$ minimization fit to extract exclusive particle yields as a function of $p_T$. An example of the simultaneous fitting is shown in Fig. \ref{fig:fitting}.

	\begin{figure}
		\centering 
		\begin{subfigure}[b]{0.49\textwidth} 
			\includegraphics[width=\textwidth]{fitting_zb.pdf}  
		\end{subfigure} 
		\begin{subfigure}[b]{0.49\textwidth} 
			\includegraphics[width=\textwidth]{fitting_zd.pdf} 
		\end{subfigure}

		\caption{ \label{fig:fitting} The result of simultaneously fitting $z^{\prime}_{\beta^{-1}}$ and $z^{\prime}_{dE/dx}$ to extract particle yields for $\pi$(red), K(blue), and p(green). The fits in the $z^{\prime}_{\beta^{-1}}$ projection are shown in the left panel while the fit results in the $z^{\prime}_{dE/dx}$ projection are shown in the right panel. The grey boxes show the active region used in the fitting procedure. }  
	\end{figure}

	In order to compare between experiments and theory it is necessary to correct for the specific detector responses that affect the measured particle yields. For instance, some of the tracks that pass through the TPC are never reconstructed by the software into a usable particle track. Additionally, some tracks are reconstructed by the TPC but not by the TOF. These effects were estimated using STARSim, an application of the widely used GEometry ANd Tracking (GEANT) library used to simulate the passage of charged particles through matter. 


\subsection{ Results}

	The primary result of this analysis is the exclusive particle yields extracted using the method outlined in Section  with the discussed corrections applied. The normalized particle yields as a function of $p_T$ are shown in Fig. \ref{fig:spectra}. It is now trivial to compute the $R_{CP}$ using the identified particle yields and the scaling factors discussed in Section \ref{sec:mc_glauber}. 

	\begin{sidewaysfigure}[ht]
		\centering 
		\includegraphics[width=\textwidth]{spectra.png} 
		\caption{ \label{fig:spectra} The corrected exclusive particle yields in the rapidity range $-0.25 \leq y \leq 0.25 $ for positive pions (top left), negative pions (bottom left), protons (top right), and anti-protons (bottom right). The spectra for each centrality is shown in a different color and scaled by successive factors of 10. } 
	\end{sidewaysfigure}



