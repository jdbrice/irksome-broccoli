\begin{figure}[H]
		\centering 
		\includegraphics[width=0.8\textwidth]{dilepton_inv_mass.png}
		\caption{ \label{fig:dilepton} The invariant mass cocktail for dielectrons in U$+$U collisions at \snn = 193 GeV. The low, intermediate and high mass regions are indicated and individual contributions to the cocktail are shown. }
	\end{figure}

\section{Future Work}
	\label{sec:future}
	Hadronic particles created in the initial formation of QGP will experience interactions with the medium throughout the system's evolution. However, leptons are largely unaffected by the QGP medium and are therefore useful for measuring characteristics of the medium throughout its space-time evolution. Opposite-sign leptons pairs (dileptons) are produced throughout the evolution of the system by various production mechanisms. Studying the production of dilepton pairs as a function of their invariant mass ($M_{ll}$) can help disentangle the thermal production of the QGP from other production mechanisms. This provides the possibility of directly measuring the local temperature of the medium at early stages in its production and evolution.

	The intermediate mass region (IMR), where $m_{\phi}$ (1.019 GeV/c$^2$) $< M_{ll} < m_{J/\psi}$ (3.096 GeV/c$^2$), has been proposed as an ideal region in which to measure the thermal production of dileptons \cite{gale_intermediate_2007}. Historically this has been a difficult measurement due to statistical limitations and a low signal to background ratio. The most significant background in this region comes from the $c\bar{c}$ and open-heavy charm production which are modified by the medium and therefore poorly modeled by simulations. 

	Recent upgrades to STAR include the Muon Telescope Detector (MTD), a dedicated muon detector, and the Heavy Flavor Tracker (HFT), a silicon based tracker in the center of the TPC\cite{yang_calibration_2014}\cite{schambach_star_2014}. The MTD will allow the dilepton invariant mass spectrum to be investigated using dimuon pairs which are less sensitive to backgrounds present in the dielectron measurements. The addition of the HFT will also provide vastly improved ability to identify the secondary decays of heavy quarks like the charm. The combination of the HFT with the MTD allow for in-medium measurement of charm contributions through e-$\mu$ correlation studies. This will help constrain the charm contributions in the IMR and improve the sensitivity to possible QGP thermal production mechanisms. I plan to produce a thorough analysis of the IMR region using the techniques outlined above as a significant portion of my PhD research.

	